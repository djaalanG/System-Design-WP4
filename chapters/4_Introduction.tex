\chapter{Introduction}
The design of a large aircraft is a complex process, comprised of numerous multi-layered steps and refinement of previously performed design iterations. Therefore, it is suitable to spread the documentation of the procedure across multiple design reports, each focused on a specific design phase. This document is the fourth, and penultimate, report in a series of five, marking the first step of the wing design phase. Its focus is on the preliminary wing design, centered around the determination of the wing loading and the stiffness design of the wing box. The design picks up from the results of the Class II Estimations and the iterated aircraft design obtained in \textit{Further Aircraft Design} \cite{Koppejan2024WorkDesign}.\\

\noindent The previous design phase, centered around the Class I Weight Estimation, wing aerodynamic design, and Class II Estimations combined with design iteration (documented in \textit{Aircraft Initial Design} \cite{Koppejan2024Aircraft1}, \textit{Wing Aerodynamic Design} \cite{Koppejan2024WingDesign}, and \textit{Further Aircraft Design} \cite{Koppejan2024WorkDesign}, respectively), resulted in a well-established aircraft geometry, as well as a list of crucial aerodynamic characteristics. These results are used to construct a numerical model able to calculate the load, stress and deflection distribution along the wing span, as well as assess the functional requirements and design options of the aircraft wingbox. The aerodynamic loading acting on the wing is obtained in \autoref{ch:ForceDiagram}. Subsequently, diagrams of the shear force, bending moment and torsion along the wingspan are constructed and analyzed for the most critical load cases. The distribution functions are used for stiffness calculations in \autoref{ch:Stiff_Calc}, which focuses on limiting the displacement and rotation of the wing tip. This is achieved by determining the distribution of the moment of inertia and torsion and constructing their respective diagrams. Both of the numerical models, along with a choice of material (case in question: AL2024-T81) allow for compiling a preliminary design of the wingbox. Three preliminary design options are derived from the functional analysis and performance requirement list for the structure (see: \autoref{ch:Prelim_WB_Des}), and tested under various manoeuvre load configurations. The most critical scenarios are identified and evaluated for aerodynamic loads.\\
    
\noindent The final wingbox configuration will be decided upon in the last of the five reports, in which a trade-off will be performed based on the derived requirements supplemented by strength calculations in tension and compression. The final design will then be evaluated for compliance with all performance requirements based on stiffness and strength. Thus, the preliminary design of the wing structure subsystem will be finalized, completing the cycle of five reports centered around the design of a large commercial aircraft.