\chapter{Summary}
This report aims at designing in detail, the wing of the aircraft. This document is the fourth report in a series of five reports describing the total design of an aircraft.\\

The first section of the report details the approach taken to calculate the variety of forces such as the shear force, bending moment and the torsion experienced by the wing. For the sake of simplicity, the complex wing design obtained in previous reports is reduced to a more manageable trapezoidal shape without sweep. The wing is modeled in XFLR5 where the wing loading is determined. From this, the shear force and bending moment can be found by integrating the wing loading one and two times respectively. Additionally, the torque distribution can is found by using the wing loading distribution along with the associated moment arms.\\

The following section is concerned with using the forces on the wing found in the previous section to design the stringer locations and number to ensure that the wing meets the required stiffness and deflection requirements.
\\

Designing the wing box using the load cases.\\


